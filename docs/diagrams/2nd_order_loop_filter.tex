\documentclass{article}
\pagestyle{empty}

\usepackage{tikz}
\usetikzlibrary{shapes,shapes.geometric,arrows}
\begin{document}

% Define some styles for use in Tikz diagrams.

\tikzstyle{simpleblock} = [draw, rectangle, minimum height=2em, minimum width=1em]

\tikzstyle{block} = [
  draw=blue!50!black!50,
  very thick,
  top color=blue!5,
  bottom color=blue!15,
  rounded corners=1.5mm,
  rectangle,
  minimum height=2em,
  minimum width=1em,
  text centered
]

\tikzstyle{mix} = [draw, circle, node distance=1cm, minimum size=0.7cm]

\tikzstyle{sum} = [draw, circle, minimum size=0.3cm, text centered]

\tikzstyle{gain} = [isosceles triangle, draw, sharp corners, scale=0.8, anchor=center]


\tikzstyle{mix} = [draw, circle, node distance=1cm, minimum size=0.7cm]
\tikzstyle{input} = [coordinate]

\begin{tikzpicture}[auto, thick, node distance=1cm, >=stealth']

  \node [input, name=input] {};
  \node [coordinate, right of=input, name=split] {};
  \node [gain, right of=split] (ki) {$k_i$};
  \node [gain, below of=ki, node distance=2.5cm] (kp) {$k_p$};
  \node [sum, right of=ki, node distance=1.5cm] (sum1) {};
  \node [coordinate, right of=sum1] (split2) {};
  \node [simpleblock, below of=split2] (z) {$z^{-1}$};
  \node [sum, right of=split2] (sum2) {};
  \node [coordinate, right of=sum2, node distance=1.5cm] (output) {};

  \draw [-] (input) -- (split);
  \draw [->] (split) -- (ki);
  \draw [->] (split) |- (kp);
  \draw [->] (ki) -- (sum1);
  \draw [-] (sum1) -- (split2);
  \draw [->] (split2) -- (z);
  \draw [->] (z) -| (sum1);
  \draw [->] (split2) -- (sum2);
  \draw [->] (sum2) -- (output);
  \draw [->] (kp) -| (sum2);

  \draw [-,thin] (sum1.north) -- (sum1.south);
  \draw [-,thin] (sum1.east) -- (sum1.west);
  \draw [-,thin] (sum2.north) -- (sum2.south);
  \draw [-,thin] (sum2.east) -- (sum2.west);

\end{tikzpicture}

\end{document}

